\chapter{Comparison of decision strategies for \hp-adaptation}
\label{app::strategies}

In this section, we oppose the adaptation behavior of all utilized strategies applied on our numerical example from Sec.~\ref{sec:errorvsperformance}, i.e.\@ \h-, \p-, and \hp-adaptation with corresponding decision strategies of error prediction and smoothness estimation by the decay of Fourier and Legendre coefficients.

We depict the distribution of finite elements after six consecutive adaptation steps, in which 30\% of cells with the highest error will be refined, and 3\% will be coarsened. For \hp-adaptation, half of all those cells will be flagged for \h- and \p-adaptation as a naive approach. In a second attempt, 90\% of cells marked for adaptation will be flagged for \p-adaptation, while the remaining 10\% will be \h-adapted. We call the second approach an educated guess for our numerical example.

The meshes of pure \h- and \p-adaptation are shown in Fig.~\ref{fig:meshnonhp}. For each \hp-adaptation strategy, grids generated with either the naive or educated guess approach are depicted in Figs.~\ref{fig:meshfourier},~\ref{fig:meshlegendre},~\ref{fig:meshprediction}.

\todo{Put pure h and p adaptation plots in two separate figure environments via minipages.}

%In every 

%We vary these with an naive

\begin{figure}
\begin{subfigure}{.49\textwidth}
  \centering
  Insert \h-adaptive mesh.
  \caption{\h-adaptive.}
\end{subfigure}
\begin{subfigure}{.49\textwidth}
  \centering
  Insert \p-adaptive mesh.
  \caption{\p-adaptive.}
\end{subfigure}
\caption{Polynomial degrees of finite elements for \h- and \p-adapted meshes.}
\label{fig:meshnonhp}
\end{figure}

\begin{figure}
\begin{subfigure}{.49\textwidth}
  \centering
  Insert \hp-adaptive Fourier naive mesh.
  \caption{Naive approach.}
\end{subfigure}
\begin{subfigure}{.49\textwidth}
  \centering
  Insert \hp-adaptive Fourier mesh.
  \caption{Educated guess approach.}
\end{subfigure}
\caption{Polynomial degrees of finite elements for \hp-adapted meshes with smoothness estimation strategy by the decay of Fourier coefficients.}
\label{fig:meshfourier}
\end{figure}

\begin{figure}
\begin{subfigure}{.49\textwidth}
  \centering
  Insert \hp-adaptive Legendre naive mesh.
  \caption{Naive approach.}
\end{subfigure}
\begin{subfigure}{.49\textwidth}
  \centering
  Insert \hp-adaptive Legendre mesh.
  \caption{Educated guess approach.}
\end{subfigure}
\caption{Polynomial degrees of finite elements for \hp-adapted meshes with smoothness estimation strategy by the decay of Legendre coefficients.}
\label{fig:meshlegendre}
\end{figure}

\begin{figure}
\begin{subfigure}{.49\textwidth}
  \centering
  Insert \hp-adaptive prediction naive mesh.
  \caption{Naive approach.}
\end{subfigure}
\begin{subfigure}{.49\textwidth}
  \centering
  Insert \hp-adaptive prediction mesh.
  \caption{Educated guess approach.}
\end{subfigure}
\caption{Polynomial degrees of finite elements for \hp-adapted meshes with error prediction strategy.}
\label{fig:meshprediction}
\end{figure}