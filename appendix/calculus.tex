\chapter{Calculus}
\label{app:calculus}
This chapter is dedicated to some useful calculus relations, which have been summarized by \textcite{bird1987}. Only those which were used in this writing will be outlined below.

%Here are some \textit{differential operations}. Here, \(s\) describes a scalar, \(\vec{v}\) and \(\vec{w}\) two respective vectors, and \( \ten{t} = \ten{t}^\intercal\) a symmetric tensor:
%\begin{align}
%  \label{ca:v_gradv}
%  \left(\vec{v} \cdot \nabla\right) \vec{v} &= \frac{1}{2} \nabla \left( \vec{v} \cdot \vec{v} \right) - \left[ \vec{v} \times \left( \nabla \times \vec{v} \right) \right] ~,
%  \\
%  \label{ca:div_sv}
%  \nabla \cdot \left( s \, \vec{v} \right)
%  &= \nabla s \cdot \vec{v} + s \left( \nabla \cdot \vec{v} \right) ~,
%  \\
%  \label{ca:grad_vw}
%  \nabla \left( \vec{v} \, \vec{w} \right)
%  &= \vec{v} \cdot \nabla \vec{w} + \vec{w} \left( \nabla \cdot \vec{v} \right) ~,
%  \\
%  \label{ca:div_vt}
%  \nabla \cdot \left( \vec{v} \cdot \ten{t} \right)
%  &= \vec{v} \cdot \left( \nabla \cdot \ten{t} \right) + \ten{t} : \nabla \vec{u} ~,
%\end{align}
%where the colon operator denotes the Frobenius inner product.
%From \ref{ca:v_gradv} applying vector triple product:
%\begin{align}
%  \label{ca:v_v_gradv}
%  \vec{v} \cdot \left[ \left( \vec{v} \cdot \nabla \right) \vec{v} \right]
%  &= \frac{1}{2} \, \vec{v} \cdot \left[ \nabla \left( \vec{v} \cdot \vec{v} \right) \right] ~.
%\end{align}

\textit{Gauss--Ostrogradskii divergence theorem}. For a vector field \(v\) in a closed domain \(V\) enclosed by a surface \(\partial V\)
\begin{align}
  \label{ca:gauss}
  \int\limits_V \left(\nabla \cdot \vec{v}\right) \differential{V}
  = \oint\limits_{\partial V} \left(\vec{n} \cdot \vec{v}\right) \differential{\partial V} ~,
\end{align}
with \(\vec{n}\) the normal vector to the boundary \(\partial V\), pointing outwards.