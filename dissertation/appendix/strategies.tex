\chapter{Comparison of decision strategies for \hp-adaptation}
\label{app::strategies}
\glsresetall

In this section, we oppose the adaptation behavior of all utilized strategies applied on our numerical example from Sec.~\ref{sec:errorvsperformance}, i.e., \h-, \p-, and \hp-adaptation with corresponding decision strategies of error prediction and smoothness estimation by the decay of Fourier and Legendre coefficients.

We depict the distribution of finite elements after six consecutive adaptation steps, in which \SI{30}{\percent} of cells with the highest error will be refined, and \SI{3}{\percent} will be coarsened. For \hp-adaptation, half of all those cells will be flagged for \h- and \p-adaptation as a naive approach. In a second attempt, \SI{90}{\percent} of cells marked for adaptation will be flagged for \p-adaptation, while the remaining \SI{10}{\percent} will be \h-adapted. We call the second approach an educated guess for our numerical example.

The meshes of pure \h- and \p-adaptation are shown in Figs.~\ref{fig:meshh},~\ref{fig:meshp}. For each \hp-adaptation strategy, grids generated with either the naive or educated guess approach are depicted in Figs.~\ref{fig:meshfourier},~\ref{fig:meshlegendre},~\ref{fig:meshprediction}.


\begin{figure}
\centering
\begin{tikzpicture}
\begin{axis}[
  scale=1.3,
  xmin=-1,xmax=1,
  ymin=-1,ymax=1,
  unit vector ratio={1 1},
  tick align=outside,
  xlabel=$x$,
  ylabel=$y$,
  colormap/OrRd,
  colorbar sampled,
  colorbar style={ylabel={finite element polynomial degree}, samples=7},
  point meta min=1.5,
  point meta max=7.5
]

\addplot graphics [
  xmin=-1,xmax=1,
  ymin=-1,ymax=1,
] {illustrations/corner-fedegrees-h-05.pdf};
\end{axis}
\end{tikzpicture}
\caption[Arrangement of finite elements with \h-adaptation.]{Arrangement of finite elements after six adaptation iterations with \h-adaptation. The colors represent different polynomial degrees $p$ of the assigned Lagrange elements $Q_p$.}
\label{fig:meshh}
\end{figure}

\begin{figure}
\centering
\begin{tikzpicture}
\begin{axis}[
  scale=1.3,
  xmin=-1,xmax=1,
  ymin=-1,ymax=1,
  unit vector ratio={1 1},
  tick align=outside,
  xlabel=$x$,
  ylabel=$y$,
  colormap/OrRd,
  colorbar sampled,
  colorbar style={ylabel={finite element polynomial degree}, samples=7},
  point meta min=1.5,
  point meta max=7.5
]

\addplot graphics [
  xmin=-1,xmax=1,
  ymin=-1,ymax=1,
] {illustrations/corner-fedegrees-p-05.pdf};
\end{axis}
\end{tikzpicture}
\caption[Arrangement of finite elements with \p-adaptation.]{Arrangement of finite elements after six adaptation iterations with \p-adaptation. The colors represent different polynomial degrees $p$ of the assigned Lagrange elements $Q_p$.}
\label{fig:meshp}
\end{figure}

\begin{figure}
\begin{subfigure}{1\textwidth}
  \centering
  \begin{tikzpicture}
\begin{axis}[
  scale=1.3,
  xmin=-1,xmax=1,
  ymin=-1,ymax=1,
  unit vector ratio={1 1},
  tick align=outside,
  xlabel=$x$,
  ylabel=$y$,
  colormap/OrRd,
  colorbar sampled,
  colorbar style={ylabel={finite element polynomial degree}, samples=7},
  point meta min=1.5,
  point meta max=7.5
]

\addplot graphics [
  xmin=-1,xmax=1,
  ymin=-1,ymax=1,
] {illustrations/corner-fedegrees-fourier-naive-05.pdf};
\end{axis}
\end{tikzpicture}
  \caption{Naive approach.}
\end{subfigure}
\vspace{1em}\\
\begin{subfigure}{1\textwidth}
  \centering
  \begin{tikzpicture}
\begin{axis}[
  scale=1.3,
  xmin=-1,xmax=1,
  ymin=-1,ymax=1,
  unit vector ratio={1 1},
  tick align=outside,
  xlabel=$x$,
  ylabel=$y$,
  colormap/OrRd,
  colorbar sampled,
  colorbar style={ylabel={finite element polynomial degree}, samples=7},
  point meta min=1.5,
  point meta max=7.5
]

\addplot graphics [
  xmin=-1,xmax=1,
  ymin=-1,ymax=1,
] {illustrations/corner-fedegrees-fourier-05.pdf};
\end{axis}
\end{tikzpicture}
  \caption{Educated guess approach.}
\end{subfigure}
\caption[Arrangement of finite elements with \hp-adaptation and the Fourier coefficient decay strategy.]{Arrangement of finite elements after six adaptation iterations with \hp-adaptation and the smoothness estimation strategy by the decay of Fourier coefficients. The colors represent different polynomial degrees $p$ of the assigned Lagrange elements $Q_p$.}
\label{fig:meshfourier}
\end{figure}

\begin{figure}
\begin{subfigure}{1\textwidth}
  \centering
  \begin{tikzpicture}
\begin{axis}[
  scale=1.3,
  xmin=-1,xmax=1,
  ymin=-1,ymax=1,
  unit vector ratio={1 1},
  tick align=outside,
  xtick={-1,-0.5,0,0.5,1},
  ytick={-1,-0.5,0,0.5,1},
  xlabel=$x$,
  ylabel=$y$,
  colormap/OrRd,
  colorbar sampled,
  colorbar style={ylabel={finite element polynomial degree}, samples=7},
  point meta min=1.5,
  point meta max=7.5
]

\addplot graphics [
  xmin=-1,xmax=1,
  ymin=-1,ymax=1,
] {illustrations/corner-fedegrees-legendre-naive-05.pdf};
\end{axis}
\end{tikzpicture}
  \caption{Naive approach.}
\end{subfigure}
\vspace{1em}\\
\begin{subfigure}{1\textwidth}
  \centering
  \begin{tikzpicture}
\begin{axis}[
  scale=1.3,
  xmin=-1,xmax=1,
  ymin=-1,ymax=1,
  unit vector ratio={1 1},
  tick align=outside,
  xtick={-1,-0.5,0,0.5,1},
  ytick={-1,-0.5,0,0.5,1},
  xlabel=$x$,
  ylabel=$y$,
  colormap/OrRd,
  colorbar sampled,
  colorbar style={ylabel={finite element polynomial degree}, samples=7},
  point meta min=1.5,
  point meta max=7.5
]

\addplot graphics [
  xmin=-1,xmax=1,
  ymin=-1,ymax=1,
] {illustrations/corner-fedegrees-legendre-05.pdf};
\end{axis}
\end{tikzpicture}
  \caption{Educated guess approach.}
\end{subfigure}
\caption[Arrangement of finite elements with \hp-adaptation and the Legendre coefficient decay strategy.]{Arrangement of finite elements after six adaptation iterations with \hp-adaptation and the smoothness estimation strategy by the decay of Legendre coefficients. The colors represent different polynomial degrees $p$ of the assigned Lagrange elements $Q_p$.}
\label{fig:meshlegendre}
\end{figure}

\begin{figure}
\begin{subfigure}{1\textwidth}
  \centering
  \begin{tikzpicture}
\begin{axis}[
  scale=1.3,
  xmin=-1,xmax=1,
  ymin=-1,ymax=1,
  unit vector ratio={1 1},
  tick align=outside,
  xtick={-1,-0.5,0,0.5,1},
  ytick={-1,-0.5,0,0.5,1},
  xlabel=$x$,
  ylabel=$y$,
  colormap/OrRd,
  colorbar sampled,
  colorbar style={ylabel={finite element polynomial degree}, samples=7},
  point meta min=1.5,
  point meta max=7.5
]

\addplot graphics [
  xmin=-1,xmax=1,
  ymin=-1,ymax=1,
] {illustrations/corner-fedegrees-prediction-naive-06.pdf};
\end{axis}
\end{tikzpicture}
  \caption{Naive approach.}
\end{subfigure}
\vspace{1em}\\
\begin{subfigure}{1\textwidth}
  \centering
  \begin{tikzpicture}
\begin{axis}[
  scale=1.3,
  xmin=-1,xmax=1,
  ymin=-1,ymax=1,
  unit vector ratio={1 1},
  tick align=outside,
  xlabel=$x$,
  ylabel=$y$,
  colormap/OrRd,
  colorbar sampled,
  colorbar style={ylabel={finite element polynomial degree}, samples=7},
  point meta min=1.5,
  point meta max=7.5
]

\addplot graphics [
  xmin=-1,xmax=1,
  ymin=-1,ymax=1,
] {illustrations/corner-fedegrees-prediction-06.pdf};
\end{axis}
\end{tikzpicture}
  \caption{Educated guess approach.}
\end{subfigure}
\caption[Arrangement of finite elements with \hp-adaptation and the error prediction strategy.]{Arrangement of finite elements after seven adaptation iterations including the initialization step with \hp-adaptation and the error prediction strategy. The colors represent different polynomial degrees $p$ of the assigned Lagrange elements $Q_p$.}
\label{fig:meshprediction}
\end{figure}