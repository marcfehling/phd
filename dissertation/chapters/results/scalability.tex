\section{\Glsfmtlong{hpc} scalability}
\label{sec:scaling}

The final part of our investigations relates to the demonstration of the scalability of our algorithms and data structures on \gls{hpc} systems, for which we will again use the JURECA supercomputer \parencite{krause2016,jureca}.

Again working with successively adapted meshes, we will measure the wall time spent in each particular section of each SOLVE-ESTIMATE-MARK-REFINE iteration, which is supposed to increase linearly with the workload determined by the number of \glspl{dof} or decrease linearly with an increasing amount of workers, i.e., number of \gls{mpi} processes. We distinguish between the following categories in each adaptation cycle similar to \textcite{bangerth2012}:
\begin{itemize}
\item \textit{Setup data structures}. Enumerate all \glspl{dof}. Determine the sparsity pattern describing locations of nonzero matrix entries. Calculate constraints for hanging nodes and boundary values. Allocate memory for all distributed data structures. Communicate between processes which matrix or vector elements they will write to that they do not own locally.
\item \textit{Assemble linear system}. Calculate the individual contribution of each locally owned cell to the global equation system. Exchange data if matrix or vector elements are stored on a different process.
\item \textit{Linear solver}. Setup both the \gls{amg} preconditioner and the conjugate gradient solver and solve the equation system in parallel.
\item \textit{Estimate error}. Calculate the error indicators on locally owned cells on basis of the current solution. Mark cells for either refinement or coarsening by computing global thresholds.
\item \textit{Estimate smoothness}. Calculate the smoothness indicators on basis of the current solution on locally owned cells marked for either refinement or coarsening. Decide whether \h- or \p-adaptation is going to be applied by computing global thresholds.
\item \textit{Coarsen and refine}. Perform coarsening and refinement and maintain the 2:1 cell balance on the \pforest{} master mesh, followed by its repartitioning. Transfer data between the outdated and updated mesh. Apply all changes made to the master mesh on the \dealii{} grid.
\end{itemize}

We will pick the parameters and features that have proven to be suitable in our numerical example. Thus, we again choose smoothness estimation by the decay of Legendre coefficients as our \hp-decision strategy. For load balancing, cell weighting is imposed proportional to the number of \glspl{dof} on each cell potentiated by the exponent $c = 1.9$.


% 'weak' scaling

To investigate scaling, we will first consider problems with increasing size solved on a fixed number of \glspl{mpi} processes, which we will realize using consecutive adaptations. We choose two different numbers of computation nodes, namely 16 and 64 nodes with 768 and 3,072 \glspl{mpi} processes in total each.

We initialize the problem with ten initial global refinements and adapt the mesh for a total of eleven iterations with the smaller amount of computing nodes, and twelve iterations for the larger one. In the chosen configuration, all available memory is used on the assigned nodes, so no more adaptation cycles are possible without running out of memory. Again to exclude the influence of the current load on the supercomputer, all runs are performed multiple times and the minimum wall time of each category is taken. This time we repeat each run seven times. The results of scaling on consecutively adapted meshes are shown in Fig.~\ref{fig:weak} up to problem sizes of 2,073,075,769 \glspl{dof}.

\begin{figure}
\begin{subfigure}{1\textwidth}
  \centering
  \begin{tikzpicture}
\begin{loglogaxis}[
  xlabel=Number of \glspl{dof},
  ylabel=Wall time {[seconds]},
  legend pos=outer north east]

% data
\addplot table [y=solve, x=ndofs, col sep=comma] {data/weak/weak-nodes16.csv};
\addlegendentry{linear solver and preconditioner};

\addplot table [y=setup, x=ndofs, col sep=comma] {data/weak/weak-nodes16.csv};
\addlegendentry{setup data structures};

\addplot table [y=assembly, x=ndofs, col sep=comma] {data/weak/weak-nodes16.csv};
\addlegendentry{assemble linear system};

\addplot table [y=compute errors, x=ndofs, col sep=comma] {data/weak/weak-nodes16.csv};
\addlegendentry{estimate error};

\addplot table [y=calculate indicators, x=ndofs, col sep=comma] {data/weak/weak-nodes16.csv};
\addlegendentry{estimate smoothness};

\addplot table [y=refine, x=ndofs, col sep=comma] {data/weak/weak-nodes16.csv};
\addlegendentry{coarsen and refine};

% optimal line
\addplot[very thick, samples=2, domain=12591105:1302365268] {10^(-7.2)*x};
\addlegendentry{optimal convergence};

% auxiliary lines
\begin{scope}
  \draw[green] ({axis cs:76800000,0}|-{rel axis cs:0,1}) -- ({axis cs:76800000,0}|-{rel axis cs:0,0});
  \draw[blue] ({axis cs:50647318,0}|-{rel axis cs:0,1}) -- ({axis cs:50647318,0}|-{rel axis cs:0,0});
\end{scope}
\addlegendimage{color=green};
\addlegendentry{$10^5$ \glspl{dof} per process};
\addlegendimage{color=blue};
\addlegendentry{each finite element in mesh};
\end{loglogaxis}
\end{tikzpicture}
  \caption{Scaling on 16 nodes or 768 \gls{mpi} processes.}
  \label{fig:weak-nodes16}
\end{subfigure} \\
\vspace{1em}
\begin{subfigure}{1\textwidth}
  \centering
  \begin{tikzpicture}
\begin{loglogaxis}[
  xlabel=Number of \glspl{dof},
  ylabel=Walltime {[seconds]}]

\addplot table [y=full cycle, x=ndofs, col sep=comma] {data/weak/weak-nodes64.csv};

\addplot[very thick, samples=2, domain=12591105:2073075769] {10^(-3)*x};

\end{loglogaxis}
\end{tikzpicture}
  \caption{Scaling on 64 nodes or 3,072 \gls{mpi} processes.}
  \label{fig:weak-nodes64}
\end{subfigure}
\vspace{-1em}
\caption[Scaling for consecutively refined meshes on different numbers of \glsfmtlong{mpi} processes.]{Scaling for consecutively refined meshes on different numbers of \glspl{mpi} processes. Each \gls{mpi} process has more than $10^5$ \glspl{dof} assigned only to the right side of the indicated vertical line. Each finite element is represented at least once in the mesh only to the right side of the designated vertical line.}
\label{fig:weak}
\end{figure}

\textcite{bangerth2012} proclaimed that linear scaling is observable in all categories if the number of \glspl{dof} per \gls{mpi} process exceeds $10^5$. We can confirm this observation in our numerical example with parallel \hp-adaptation as well.

During the first few adaptation cycles in our application, the wall time attributed to the solution category shows a major increase which is more than just linear. After six adaptation cycles, i.e., right of the indicated vertical line in Fig.~\ref{fig:weak}, each reference finite element from the collection will be assigned to at least one cell due to the way we configured the scenario, and the aforementioned curve flattens and increases only linearly as expected.

We observe a similar behavior in the number of iterations that the linear solver requires in Fig.~\ref{fig:solveriterations}. Here, the number of iterations first increases exponentially and stagnates after said number of adaptation cycles.

\begin{figure}
\centering
\begin{tikzpicture}
\begin{axis}[
  scale=1.15,
  xlabel={Adaptation cycle},
  ylabel={Number of solver iterations},
  grid=major,
  legend columns=3,
  legend cell align=left,
  legend style={at={(0.39,1.02)}, anchor=south, /tikz/every even column/.append style={column sep=0.5cm}}]

\addplot table [y=iterations, x=cycle, col sep=comma] {data/weak/weak-nodes16.csv};
\addlegendentry{16 nodes};

\addplot table [y=iterations, x=cycle, col sep=comma] {data/weak/weak-nodes64.csv};
\addlegendentry{64 nodes};

\begin{scope}
  \draw[blue, very thick] ({axis cs:5,0}|-{rel axis cs:0,1}) -- ({axis cs:5,0}|-{rel axis cs:0,0});
\end{scope}
\addlegendimage{color=blue};
\addlegendentry{each reference finite element in mesh};
\end{axis}
\end{tikzpicture}
\caption[Number of solver iterations at different cycles of consecutive adaptations.]{Number of solver iterations at different cycles of consecutive adaptations.}
\label{fig:solveriterations}
\end{figure}

%We suspect that the randomness of the distribution of 

%We suspect that the mixture of many different finite elements are the reason for this behavior

%that adding finite elements of a higher order than before results in a higher complexity in the matrix, that is responsible for a longer solution.
We suspect that the rather heterogeneous association of the finite elements by the decision algorithms
%has a negative influence on the conditioning of the system matrix and
has a similar effect on the distribution of nonzero entries in the system matrix, for which \gls{amg} preconditioners are not designed for. It appears that we could make use of a more suitable preconditioner. Although it was the best option at our disposal at the time of this dissertation, we may think about an alternative to this for future applications.

\textcite{mitchell2010} presented \hp-multigrid methods for sequential applications. They combined multilevel methods on a geometric hierarchy with those on a hierarchy of finite elements with different polynomial degrees $p$. The \hp-adaptive domain will be first relaxed by successively decreasing the polynomial degree of all associated finite elements until only linear ones remain. Next, the usual geometric multigrid method is applied, while results are interpolated back to finite elements of the original polynomial degrees as a last step. The embedding of \p-relaxation and \p-interpolation in either \gls{amg} or \gls{gmg} preconditioning is promising and will be the subject of future work. Different combinations of multilevel methods in a polynomial, geometric, and algebraic sense are possible, as \textcite{fehn2019} demonstrated on static non-adaptive meshes. % for matrix-free \gls{dg} methods.
%Instead of relying on \gls{amg} methods, \gls{gmg} preconditioners are expected to work more robust and should be the method of choice.
%\Gls{gmg} methods for \hp-adaptive refinement in serial applications have already been developed by \textcite{mitchell2010}. \textcite{clevenger2019} worked out \gls{gmg} preconditioners for parallel \h-adaptive \gls{fem} for either \gls{cg} or \gls{dg} methods and made their algorithm available in the \dealii{} library. The development of a corresponding preconditioner for parallel \hp-adaptive \gls{fem} will be subject of future work, as well as its application and comparison with the \gls{amg} equivalents.
Therefore, an even more favorable error to wall time performance of \hp- compared to \h-adaptive methods are conceivable than presented in Fig.~\ref{fig:errorwalltime} and described in Sec.~\ref{sec:errorvsperformance}.

Furthermore, we find a similar behavior of the wall time in the assembly category.
%A higher polynomial degree of the finite element results in a larger amount of \glspl{dof} that couple with each other, as already pointed out in Sec.~\ref{sec:balancing}.
%Due to our choice of adaptation strategies, we aggressively apply \p-refinement
%in the beginning of the application,
%which explains the rate of increase of the wall time at its early stage.
%which we attribute to the additional workload imposed by the corresponding higher quadrature rules.
%Since we aggressively apply \p-adaptation, this is the reason for the increase of runtime.
The evaluation of finite element shape functions and their derivatives on all quadrature points is an expensive operation, even if it is performed on the reference cell to be projected onto an actual cell by mapping. Furthermore in the context of \hp-adaptive methods, this evaluation has to be performed for every single reference finite element in our collection.
%is expensive, we need to evaluate finite elements on all quadrature points an nodes on the reference cells and calculate integrals after mapping happened.
%In the context of \hp-adaptive methods, we will only 
Specifically for \dealii{}, these values will only be evaluated whenever a cell with the corresponding reference finite element assigned is visited for the first time. Thus, not all of these evaluation objects are calculated until all finite elements are actually represented in the domain.
%Thus with our choice of an aggressive strategy in favor of \p-adaptation, not all of these evaluation objects are created until all finite elements are actually represented in the domain. In fact in our application, we even evaluate all finite elements over again.
In our application, we unintentionally re-calculated these values again in each adaptation step. Combined with our choice of a decision strategy in favor of \p-adaptation, we observe a cohere increase of the wall time until every reference finite element is represented at least once in the domain. For future applications we will move the evaluation of all reference finite element in front of the actual timing investigations.


% strong scaling

For strong scaling, problems are set to a fixed size and are solved with an increasing amount of \gls{mpi} processes. This time, we just solve one individual adaptation cycle on a tailored mesh, that has been prepared from a previous run via serialization.
%, we prepare a tailored mesh for these considerations, and solve that particular cycle with a varying number of processors.

To prepare these meshes, we consider two different scenarios which will be constructed as follows: A smaller scenario is initialized with ten global refinements, and a larger one with twelve refinements. Both will be adapted successively in six adaptation cycles, which results in each reference finite element being represented at least once in the whole domain. This leads to number of \glspl{dof} of 50,736,415 and 969,257,276 in total for the respective scenarios.

With serialization, both problems will be solved at their advanced stage with varying amounts of \gls{mpi} processes, and the wall times of each section in the program will be recorded. We again repeat each run for a total of seven times and take the minimum wall time in each category, except for the largest run in order to solve the bigger problem on 1,024 nodes or 49,152 \gls{mpi} processes, which we only repeated five times. % before we completely exhausted our entire computing time quota.

The results of strong scaling are shown in Fig.~\ref{fig:strong}.
Again, we identify linear scaling whenever the number of \glspl{dof} per \gls{mpi} process exceeds $10^5$, which again coincides with the observations of \textcite{bangerth2012}.

\begin{figure}
\begin{subfigure}{1\textwidth}
  \centering
  \begin{tikzpicture}
\begin{loglogaxis}[
  xlabel=Number of \gls{mpi} processes,
  ylabel=Walltime {[seconds]}]

\addplot table [y=full cycle, x=ncpus, col sep=comma] {data/strong/strong-nrefs10_withoutlarge.csv};

\addplot[very thick, samples=2, domain=48:6144] {10^(3)*x^(-1)};

\end{loglogaxis}
\end{tikzpicture}
  \caption{Strong scaling for a fixed problem size of roughly 51 million \glspl{dof}.}
  \label{fig:strong-nrefs10}
\end{subfigure} \\
\vspace{1em}
\begin{subfigure}{1\textwidth}
  \centering
  \begin{tikzpicture}
\begin{loglogaxis}[
  xlabel=Number of \gls{mpi} processes,
  ylabel=Wall time {[seconds]},
  legend pos=outer north east]

% data
\addplot table [y=solve, x=ncpus, col sep=comma] {data/strong/strong-nrefs12.csv};
\addlegendentry{linear solver and preconditioner};

\addplot table [y=setup, x=ncpus, col sep=comma] {data/strong/strong-nrefs12.csv};
\addlegendentry{setup data structures};

\addplot table [y=assembly, x=ncpus, col sep=comma] {data/strong/strong-nrefs12.csv};
\addlegendentry{assemble linear system};

\addplot table [y=compute errors, x=ncpus, col sep=comma] {data/strong/strong-nrefs12.csv};
\addlegendentry{estimate error};

\addplot table [y=calculate indicators, x=ncpus, col sep=comma] {data/strong/strong-nrefs12.csv};
\addlegendentry{estimate smoothness};

\addplot table [y=refine, x=ncpus, col sep=comma] {data/strong/strong-nrefs12.csv};
\addlegendentry{coarsen and refine};

% optimal line
\addplot[very thick, samples=2, domain=768:49152] {10^(5)*x^(-1)};
\addlegendentry{optimal convergence};

% auxiliary lines
\begin{scope}
  \draw[green] ({axis cs:9692.57276,0}|-{rel axis cs:0,1}) -- ({axis cs:9692.57276,0}|-{rel axis cs:0,0});
\end{scope}
\addlegendimage{color=green};
\addlegendentry{$10^5$ \glspl{dof} per process};
\end{loglogaxis}
\end{tikzpicture}
  \caption{Strong scaling for a fixed problem size of roughly 970 million \glspl{dof}.}
  \label{fig:strong-nrefs12}
\end{subfigure}
\caption[Strong scaling for one advanced adaptation cycle at different problem sizes.]{Strong scaling for one advanced adaptation cycle at different problem sizes. Each \gls{mpi} process has more than $10^5$ \glspl{dof} assigned only to the left side of the indicated vertical line.}
\label{fig:strong}
\end{figure}