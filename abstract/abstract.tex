\begin{abstract}
Efficient algorithms for the numerical solution of partial differential equations are required to solve problems on an economically viable timescale.
In general, this is achievable by adapting the resolution of the discretization to the investigated problem, as well as exploiting hardware specifications.
For the latter category, parallelization plays a major role for modern multi-core and multi-node architectures, especially in the context of high-performance computing.
%, which plays a major role on multi-core and multi-node in the context of high performance computing.
%i.e.\@ multicore s with parallelization, especially in the context of high performance computing.
%Implementation is difficult, but we ant to make that public.
%Generic algorithms

Using finite element methods, solutions are approximated by discretizing the function space of the problem with piecewise polynomials. With \hp-adaptive methods, the polynomial degrees of these basis functions may vary on locally refined meshes.

%Finite element methods discretize the function space with piecewise polynomials, in which we approxi

% in which we are supposed to find our solution, which can be ... with \hp-adaptation 

We present algorithms and data structures required for generic \hp-adaptive finite element software applicable for both continuous and discontinuous Galerkin methods on distributed memory systems. %relying on the Message Passing Interface (MPI).
%Both static and dynamic meshes will be considered, thus the resolution is allowed to change in the process.
%We allow both static meshes with a prescribed resolution as well as dynamic ones which change it during the coarse of it.
Both function space and mesh may be adapted dynamically during the solution process.

%Static
%We present nontrivial stuff.
We cover details concerning the unique enumeration of degrees of freedom with continuous Galerkin methods, the communication of variable-sized data, and load balancing.
%and load balancing on distributed memory systems, and
%load balancing on distributed memory systems
%Dynamic
%more general,
%Implementation details will be covered concerning
%We focus on nontrivial/
%The most diifcult parts
%Further, we present algorithms for the automatic identification of areas for adaptation and decision criterion to decide between \h- or \p-adaptation. 
Further, we present strategies to determine the type of adaptation on basis of error prediction and smoothness estimation via the decay rate of coefficients of Fourier and Legendre series expansions. Both refinement and coarsening are considered.

%we present methods to decide which type of adaptation to impose on bassis of
%refinement and coarsening strategies for combined \hp-adaptation
%error prediction
%smoothness estimation via the decay of coefficients of Fourier and Legendre series expansion

A reference implementation in the open-source library \dealii{}\footnote{\url{https://www.dealii.org}} is provided
%using \pforest{}\footnote{\url{http://p4est.github.io}} as a mesh oracle.
%As a demonstration, we 
and applied to the Laplace problem on a domain with a reentrant corner which invokes a singularity. With this example, we demonstrate the benefits of the \hp{}-adaptive methods in terms of error convergence and show that our algorithm scales up to 49,152 \texttt{MPI} processes.
\end{abstract}