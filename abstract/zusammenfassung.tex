\begin{abstract}
Die Urbanisierung unserer Gesellschaft und die dadurch steigende Auslastung der Verkehrsinfrastruktur erfordert eine kontinuierliche Weiterentwicklung bestehender Brand-schutz- und Evakuierungskonzepte zur Gewährleistung der Personensicherheit. Je nach Komplexität des zu betrachtenden Objekts ist die Auslegung anhand von statischen Richtlinien im Allgemeinen nicht hinreichend, so dass angepasste, individualisierte Betrachtungen und Bewertungen vorgenommen werden müssen. Zunehmend können diese nur über Simulationen erfolgen, die durch wachsende Ansprüche an Vorhersagegenauigkeiten mehr und mehr Rechenbedarf benötigen.

Eine intelligente Verteilung der Rechenressourcen ließe den Zeit- und somit auch Kostenaufwand für solche Simulationen reduzieren, oder anders formuliert, würde bei gleicher Rechenzeit ein genaueres Ergebnis liefern. Dies ist sowohl durch eine Lastenverteilung auf mehrere Prozessoren (Parallelisierung), als auch durch eine adaptive Auflösung der Simulation mit lokaler Verfeinerung möglich, die aufgrund der Dynamik einer Simulation nicht \textit{a priori}, sondern progressiv erfolgen muss. Ziel dieses Promotionsvorhabens ist die Entwicklung solcher neuen, effizienten Algorithmen.

Durch ihre vielversprechenden Adaptionsmöglichkeiten fiel die Wahl auf die Finite-Elemente-Methode (FEM) zur Raumdiskretisierung. Diese erlaubt die dynamische Festlegung sowohl der lokalen Gitterauflösung (h-Adaptivität), als auch des Polynomgrads des Funktionenraums einer jeden Zelle (p-Adaptivität), oder in anderen Worten der Genauigkeit der Lösung innerhalb einer Zelle. In Kombination mit Parallelisierung ist dieses Verfahren bereits in der wissenschaftlichen Gemeinschaft etabliert, aber bisher nur für einige Sonderfälle implementiert. In dieser Doktorarbeit wird eine dimensionsunabhängige, generalisierte Formulierung eines parallelen, hp-adaptiven Finite-Elemente-Algorithmus präsentiert, die ergänzt durch eine effiziente Lastenverteilung eine Skalierung auf Großrechnern mit tausenden von Rechenkernen ermöglicht. Eine Beispielimplementation erfolgt in der \dealii{}\footnote{\url{https://www.dealii.org}} Softwarebibliothek. Die Bibliothek kann bei der Erstellung von Simulationssoftware mit der Finite-Elemente-Methode unterstützend hinzugezogen werden, und zwar unabhängig vom tatsächlichen Anwendungsbereich, wodurch das Ergebnis dieser Arbeit auch anderen Wissenschaftsbereichen zugänglich sein wird.

Letztlich wird mit eben dieser Bibliothek ein Strömungslöser als Machbarkeitskonzept entwickelt, der die Rauchdynamik mit diesem neuen Algorithmus abbilden kann. Dies umfasst auch die Wahl geeigneter Entscheidungskriterien für die adaptiven Verfahren. Zur Verifikation und Validierung des Programms werden ausgewählte analytische Testfälle und Experimente verwendet, auch um die Eignung der massiv parallelen, hp-adaptiven Finiten-Elemente-Methode für Anwendungen im Brandschutz festzustellen und die Nutzung von Supercomputern in der zivilen Sicherheitsforschung zu etablieren.
\end{abstract}