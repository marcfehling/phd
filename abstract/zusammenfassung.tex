\begin{abstract}
Für die numerische Lösung partieller Differentialgleichungen sind effiziente Algorithmen erforderlich, um Probleme auf einer wirtschaftlich tragbaren Zeitskala zu lösen.
Im Allgemeinen ist dies durch die Anpassung der Diskretisierungsauflösung an das untersuchte Problem sowie durch die Ausnutzung der Hardwarespezifikationen möglich.
Für die letztere Kategorie spielt die Parallelisierung eine große Rolle für moderne Mehrkern- und Mehrknotenarchitekturen, insbesondere im Kontext des Hochleistungsrechnens.

Mit Hilfe von Finite-Elemente-Methoden werden Lösungen durch Diskretisierung des assoziierten Funktionsraums mit stückweisen Polynomen approximiert. Bei \hp-adaptiven Methoden können die Polynomgrade dieser Basisfunktionen auf lokal verfeinerten Gittern variieren.

In dieser Dissertation werden Algorithmen und Datenstrukturen vorgestellt, die für generische \hp-adaptive Finite-Elemente-Software benötigt werden und sowohl für kontinuierliche als auch diskontinuierliche Galerkin-Verfahren auf Systemen mit verteiltem Speicher anwendbar sind. Sowohl der Funktionsraum als auch das Gitter können während des Lösungs-prozesses dynamisch angepasst werden.

Im Besonderen erläutert werden die eindeutige Nummerierung von Freiheitsgraden mit kontinuierlichen Galerkin-Verfahren, die Kommunikation von Daten variabler Größe und die Lastenverteilung.
Außerdem werden Strategien zur Bestimmung des Adaptierungstyps auf der Grundlage von Fehlerprognosen sowie Glattheitsschätzungen vorgestellt, die über die Zerfallsrate von Koeffizienten aus Reihenentwicklungen nach Fourier und Legendre bestimmt werden. Dabei werden sowohl Verfeinerung als auch Vergröberung berücksichtigt.

Eine Referenzimplementierung erfolgt in der Open-Source-Bibliothek \dealii{}\footnote{\url{https://www.dealii.org}} und wird auf das Laplace-Problem auf einem Gebiet mit einer einschneidenden Ecke angewandt, die eine Singularität aufweist. Anhand dieses Beispiels werden die Vorteile der \hp{}-adaptiven Methoden hinsichtlich der Fehlerkonvergenz und die Skalierbarkeit der präsentierten Algorithmen auf bis zu 49.152 \texttt{MPI}-Prozessen demonstriert.
\end{abstract}