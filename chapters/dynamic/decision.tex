\section{Decision criteria}
\label{sec:decision}

A common observation is that increasing the grid resolution or the polynomial degree of the basis functions will decrease the difference between the finite element approximation $u_\text{hp}$ and the actual solution $u$.

In fact, the impact of these adaptation techniques on this error is well understood. \textcite[Thm.~3.4]{babuska1990} determined an upper bound for the error that depends both on the cell diameter $h$ and the polynomial degree $p$:
\begin{align}
\label{eq:errorbound_hp} \left\|e_\text{hp}\right\|_{H^{1}(\Omega)} &\leq C \, h^{\mu} \, p^{-(m-1)} \, \|u\|_{H^{m}(\Omega)} \,\text{,}
\end{align}
where $e_{hp} = u - u_\text{hp}$ denotes the error function, $m$ is a measure for the regularity of the solution $u$, $C$ is a constant dependent on $m$, and $\mu = \min \left(p, m - 1\right)$.

These modifications do not have to happen uniformly on a global scale, but can be applied locally, since the global error consists of the local ones of each cell $K$:
\begin{align}
\label{eq:error_sum} \left\|e_\text{hp}\right\|_{H^1(\Omega)}^2 = \sum\limits_{K \in \Omega} \left\|e_\text{hp}\right\|_{H^1(K)}^2 \,\text{.}
\end{align}
Thus it all comes down to find those sections that have a significant contribution to the global error, and mitigate their impact by local adaptation. \textcite[Thm.~5.1]{guo1986} predicted exponential convergence with the number of \glspl{dof} $n_\text{dofs}$ on a suitable \hp-adapted mesh:
\begin{align}
\label{eq:errorbound_exp} \left\|e_\text{hp}\right\|_{H^{1}(\Omega)} &\leq C \, \exp\left(- b \, n_\text{dofs}^{1 / 3}\right) \,\text{,}
\end{align}
where constants $b > 0$ and $C$ are both independent of $n_\text{dofs}$.

With sufficient information about the investigated scenario, an \hp-adaptive grid can be tailored to its specifics manually. However, grid adjustments by hand may not be optimal. Furthermore, not all peculiarities about the scenario are generally known in advance, which is especially the case for problems with complex geometries and time dependent ones.

Hence we need to elaborate on algorithms to automatically decide which subsets of the domain qualify for adaptation. With this technique, we typically set up a coarse mesh along with basis functions of a low polynomial degree and obtain a tailored mesh after several adaptation iterations.

In this section, we present different ways to identify areas whose adaptation will be most profitable, and to choose the most beneficial type of adaptation. For hp-adaptation in particular, \textcite{mitchell2014} reviewed and compared a selection of strategies in detail. We demonstrate a subset of their recommendations in terms of performance and applicability, i.e.\@ those strategies that only require locally relevant part of the current solution.



\subsection{Adaptation strategies}
\label{ssec:strategy}

We will decide on the basis of adaptation criteria for every cell which ones will be considered to be adapted. Typical criteria are errors or their estimates, but also predicted errors or smoothness indicators are used, which will be presented in the following sections. \textcite[Sec.~5.2]{bangerth2003} describe different strategies on how to decide based on these adaptation criteria, from which we present a commonly used selection.

So called \textit{fixed-error-reduction} or \textit{fixed-fraction} strategies select subsets of cells whose criteria accumulate to a predefined fraction of their global sum. This strategy is only applicable when the sum of all criteria actually has meaning, for example local errors which add up to the global one. Further, it may lead to optimal meshes for several problems, but tends to only adapt very few cells whenever singularities are encountered.

On the other hand, strategies known as \textit{fixed-rate} or \textit{fixed-number} pick predefined fractions of cells with the lowest or highest criteria for adaptation. This allows to predict the growth of cells, but may not lead to an optimal mesh since more cells may be adapted than necessary.

For either strategy, when using actual errors or error indicators as adaptation criteria, we typically select the subset of cells corresponding to the higher error for refinement, while the subset with the lower error is considered for coarsening.

Applicable implementations of these strategies involve binary searches to determine the section of cells relevant for adaptation. For parallel computations, according algorithms have been developed by \textcites[Sec.~3.1]{burstedde2008}[Sec.~5.1]{bangerth2012}.



\subsection{Error estimation}
\label{ssec:estimation}

The determination of the error for our finite element approximation requires the actual solution to be at our disposal. However, this is not the case in general, and we need to come up with an alternative measure.

\textcite{kelly1983} worked out an \textit{a posteriori} error estimator for the generalized Poisson equation $-\nabla \cdot \left( a \nabla u \right) = f$, where $a$ is a function usually describing material characteristics. They determined an upper bound $\eta_K$ for the error on each cell by balancing the gradient of the finite element approximation $u_\text{hp}$ on all faces $F$ of the cell's boundary:
\begin{align}
\label{eq:kelly} \|e_\text{hp}\|_{H_1(\Omega)}^2 &\leq C \sum\limits_{K \in \Omega} \eta_K^2 &&\text{with}&  \eta_K^2 &= \sum\limits_{F \in \partial K} c_F \int\limits_{F} \left[ a \, \frac{\partial u_\text{hp}}{\partial n} \right]^2 \differential{o} \,\text{,}
\end{align}
where $C$ is independent of the solution, but depends on $a$, and
\begin{align*}
\left[ a \, \frac{\partial u_\text{hp}}{\partial n} \right] = \left. a \, \frac{\partial u_\text{hp}}{\partial n_K} \right|_K + \left. a \, \frac{\partial u_\text{hp}}{\partial n_J}\right|_J
\end{align*}
denotes the jump of the approximation's gradient on the face between two adjacent cells $K$ and $J$. Hence \textcite{ainsworth1997a} attribute this estimator to the class of gradient recovery estimators.

The constant $c_F$ depends on the characteristics of each individual face of the cell. \textcite{kelly1983} originally used the constant $c_F = \frac{h_K}{24 \, a_\text{min} \, p_K}$ on each face, on which we determine the minimum $a_\text{min}$ of the given function. Here, $h_K$ and $p_K$ denote both cell diameter and polynomial degree of the finite element on cell $K$, respectively. \textcite{davydov2017} proposed a different constant for \hp-adaptive \gls{fem}: They recommend $c_F = \frac{h_F}{2 \, a_\text{min} \, p_F}$ with $h_F$ the face diagonal and $p_F = \max\left(p_K, p_J\right)$ the largest polynomial degree of adjacent elements $K$ and $J$ on this particular face.

This estimator has been worked out for the Poisson equation, but has proven its applicability on other problems as well, where this is no longer meant to be an estimator, but rather an error indicator\todo{cite deal.II kelly error estimator website}.

The error estimator is already implemented in \dealii{}. \todo{cite deal.II kelly error estimator website}

We will use these error estimates to decide w. We are still left to decide which type of adaptation we want to apply, i.e.\@ \h-adaptation or \p-adaptation.



\subsection{Error prediction}
\label{ssec:prediction}

\cite{babuska1990} determined upper error bounds for numerical solutions based the distribution of finite elements. Both mesh resolution and polynomial degrees of the basis functions have a different, yet quantifiable influence on the error leading to Eq.~(\ref{eq:errorbound_hp}).

Their findings are valid not only for the numerical solution on a global scope, but on subsets of the domain as well. Local changes by \h- and \p-adaptation will thus result in different local error bounds. This motivates a strategy to locally decide which type of adaptation to impose based on the refinement history which has been proposed by \textcite{melenk2001}: We can predict how the current error will change whenever certain areas of our domain are considered for adaptation in the following iteration. These predicted error estimates allow us to decide whether the choice of adaptation in the previous step was justified, and provide the foundation for it on the next one.

We determine how the error bounds on two different distributions of finite elements will change by calculating their ratio. For this, we assume that both the actual error and its upper bound change with the same rate, which allows us to equate both ratios. We further assume that the solution is sufficiently regular ($m \gg p$). The ratio of errors then reads:
\begin{align}
\label{eq:errorratio_hp} \frac{||e_{h_f p_f}||_{H^{1}(K)}}{||e_{h_a p_a}||_{H^{1}(K)}} = \frac{h_f^{p_f}}{h_a^{p_a}} \, \left(\frac{p_f}{p_a}\right)^{-(m-1)} \,\text{,}
\end{align}
where subscripts $a$ and $f$ denote the finite element that is currently active or will be active after adaptation, respectively.

If we only consider \h-adaptation and leave the polynomial degree of the basis function unchanged ($p_f = p_a \equiv p$), we end up with the classical error bound \todo{cite}:
\begin{align}
\label{eq:errorratio_h} \frac{||e_{h_f p}||_{H^{1}(K)}}{||e_{h_a p}||_{H^{1}(K)}} = \left( \frac{h_f}{h_a} \right)^p \,\text{.}
\end{align}

However, if only \p-adaptation is considered and we keep the domain unchanged ($h_f = h_a \equiv h$), the ratio of errors still depends on the regularity of the actual solution which is not at our disposal in general.
\begin{align}
\label{eq:errorratio_p} \frac{||e_{h p_f}||_{H^{1}(K)}}{||e_{h p_a}||_{H^{1}(K)}} = h^{p_f - p_a} \, \left(\frac{p_f}{p_a}\right)^{-(m-1)}
\end{align}
Following the considerations of \cite{melenk2001}, we expect \p-adaptation to change the error exponentially with the increment of the polynomial degree
\begin{align}
\label{eq:errorratio_p_exp} \frac{||e_{h p_f}||_{H^{1}(K)}}{||e_{h p_a}||_{H^{1}(K)}} \simeq c^{p_f - p_a} \,\text{,}
\end{align}
where $c$ is a constant independent of the cell diameter $h$.

We suggest a similar approach for the hp-adaptation case as well. The above ratio assumes that the underlying mesh has not been changed. We thus identify Eq.~(\ref{eq:errorratio_p_exp}) with an unaltered cell diameter ($h \equiv h_a$) in Eq.~(\ref{eq:errorratio_hp}) resulting in:
\begin{align}
\label{eq:errorratio_hp_exp} \frac{||e_{h_f p_f}||_{H^{1}(K)}}{||e_{h_a p_a}||_{H^{1}(K)}} \simeq \left( \frac{h_f}{h_a} \right)^{p_f} \, c^{p_f - p_a} \,\text{.}
\end{align}
We will take this factor in all following considerations to predict how errors will change.

% ----------

% underscore
Again division on multiple cells in case of h-refinement.

For h-coarsening or hp-coarsening, we need to take special care. And h-coarsening involves summing up. Note that we take the new polynomial degree here.
\begin{align}
Some other nice formula hp coarsening
\end{align}
For pure h coarsening we consider
\begin{align}
Some other nice formula h coarsening
\end{align}

% start from here

All above relations not only hold for the whole domain $\Omega$, but also qualify for local adaptation on subsets of it, i.e.\@ any cell $K \in \Omega$. We write introduce control parameters

we need to subdivide it into it's children, thus

Isotropic refinement with quadrilaterals in two and hexahedrals in three dimensions.

We take
Nothing changes
\begin{align}
\label{eq:nothing} ||e_{h p}||_{H^1(K)} \simeq \gamma_n \, ||e_{h p}||_{H^1(K)}
\end{align}
with a control parameter \(\gamma_n = 1\).

p adaptation
\begin{align}
\label{eq:p_adaptation} ||e_{h p_f}||_{H^1(K)} \simeq \gamma_p^{p_f-p_a} \, ||e_{h p_a}||_{H^1(K)}
\end{align}
with a control parameter \(\gamma_p = \sqrt{0.1}\) chosen to be in accordance with \cite{melenk2001} and \cite{mitchell2014}.

hp refinement
\begin{align}
\label{eq:hp_refinement} ||e_{h_f p_f}||_{H^1(K_c)} \simeq \gamma_h \, n_c^{-1} \, 0.5^{p_f} \, \gamma_p^{p_f - p_{a_c}} ||e_{h_a p_a}||_{H^1(K)}
\end{align}
with control parameter \(\gamma_h = 1\).

hp coarsening
\begin{align}
\label{eq:hp_coarsening} ||e_{h_f p_f}||_{H^1(K)} \simeq \sum\limits_{c} \gamma_h^{-1} \, 2^{p_f} \, \gamma_p^{p_f - p_{\text{a}_c}} ||e_{h_{a_c} p_{a_c}}||_{H^1(K_c)}
\end{align}

% ---------

We now have an algorithm to predict the error for the next adaptation step on basis of the current one. We are left to find a suitable criterion on how to use it to actually decide .

The original idea of \cite{melenk2001} was to compare  for \h-refinement.
The idea behind this is that

We enhanced this approach for \h-coarsening
Similarly for coarsening, we can do this.

Another approach would be to treat the difference of predicted and estimated error as an indicator for each cell. From all cells that would be refined, we will consider those with the greatest values for p-refinement and the others for h-refinement. Analougously, those cells with the least difference will be \p-coarsened, while the

The error predictor has been implemented in the \dealii library as part of this dissertation. A development log can be found here.



\subsection{Smoothness estimation}
\label{ssec:smoothness}

Text.

\cite{mavriplis1994} introduced with the Legendre coefficient decay..


\cite{bangerth2009} introduced Fourier coefficient decay.