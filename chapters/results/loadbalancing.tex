\section{Load balancing heuristics}
\label{sec:heuristics}

% introduction

For parallel computations on distributed memory systems, the global domain is partitioned into several subdomains, each of which is assigned to a single process. Such a mesh decomposition is showcased in Fig.~\ref{fig:decomposition}.

\begin{figure}
\centering
Insert decomposition vtk plot here.
\caption{Decomposition of the mesh from Fig.~\ref{fig:fedegrees} on 16 \gls{mpi} processes with constant weighting. Each color represents a different subdomain.}
\label{fig:decomposition}
\end{figure}

Proper load balancing is necessary for an efficient use of all computational resources. Especially on \gls{hpc} systems with lots of available processors, this is a critical feature. For \hp-adaptive \gls{fem}, we presented approaches for load balancing in Sec.~\ref{sec:loadbalancing} by assigning weights to each individual cell. We relate the weight to the number of \glspl{dof} on each cell potentiated by an exponent that we will determine in the upcoming investigations. Or in other words, cell weights are chosen proportional to $n_\text{dofs}^i$ with the exponent $i$ to be ascertained.

Although all three \hp-adaptive strategies have demonstrated a similar performance as shown in Sec.~\ref{sec:errorvsperformance}, we pick only one adaptation strategy for our parallel investigations. We choose the smoothness estimation strategy based on the decay of Legendre coefficients as the most efficient one for this purpose.

% system specifications

Investigations are carried out on the JURECA supercomputer \parencite{krause2016}. Each computing node is equipped with two Intel Xeon E5-2680 v3 processors with twelve cores running at 2.5GHz and 128GB of memory. With simultaneous multithreading, a total of 48 processes are available per node. Communication between nodes happens via a Mellanox EDR InfiniBand high-speed network.

Our investigations are performed on two distinct nodes, which provides a total of 96 \gls{mpi} processes and involves communication between two physically independent memory segments. We expect that this setup yields representative results that can be extrapolate on even larger problem sizes.

For our investigations, we use a 'flat' \gls{mpi} model: Every thread will be assigned to an individual \gls{mpi} process and no additional thread parallelization is invoked. Although \dealii{} provides such a feature via \gls{tbb}, we refrain from using it to measure the pure \gls{mpi} performance for all analyses in this dissertation.

% problem specifications

To qualify our problem for parallel computations, we need to increase its size drastically. The problem is initialized with \todo{12?} global refinements and gets adapted by \todo{6?} iterations. For the strategy with the Legendre coefficient decay, this results in a total number of \todo{1,000,000?} \glspl{dof}, so that each process will be assigned with a number of \glspl{dof} in the order of \todo{100,000}. Each type of finite element from the provided collection is represented at least once in the mesh.

This advanced scenario will form the basis of our investigations to see how different weighting exponents affect the wall time, and which one provides a minimum. With serialization, we ensure that each of these runs conforms to the same conditions. Again, to mitigate the impact of temporary slowdowns on the supercomputer due to high loads on memory and network bandwidth, we repeat each run for a total of \todo{five?} times and take the minimum wall time in each category over all runs.

% results

For varying weighting exponents, we compare the wall times of the full adaptation cycle and its relevant sections
%For resumed scenarios in which the mesh will be partitioned according to weights with different weighting exponents, we compare the wall time of critical sections of the program as well as the total wall time
%between runs with different weighting exponents
in Fig.~\ref{fig:weights}.

\todo{Second scale for assembly/discontinued axis with groupplot https://tex.stackexchange.com/questions/46422/axis-break-in-pgfplots}
\begin{figure}
\centering
\begin{tikzpicture}
\begin{axis}[
  xlabel=Weighting exponent,
  ylabel=Wall time {[seconds]},
  legend pos=outer north east]

\addplot table [y=full cycle, x=weighting exponent, col sep=comma] {data/weight/weight.csv};
\addlegendentry{full cycle};

\addplot table [y=assembly, x=weighting exponent, col sep=comma] {data/weight/weight.csv};
\addlegendentry{assemble linear system};

\addplot table [y=solve, x=weighting exponent, col sep=comma] {data/weight/weight.csv};
\addlegendentry{linear solver and preconditioner};
\end{axis}
\end{tikzpicture}
\caption[Wall times for load balancing with varying weighting exponents.]{Wall times of a complete adaptation cycle and its relevant parts for load balancing. Weights proportional to $n_\text{dofs}^i$ will be assigned to each cell with varying exponents $i$.}
\label{fig:weights}
\end{figure}

The assembly of the equation system and the its solution are identified as the critical sections whose wall time is affected by the number of \glspl{dof}. We see that the solution of the equation system takes about \todo{90}\% of the total wall time and is the crucial factor for proper load balancing. The minimal wall time for both solver and the full cycle is reached with a weighting exponent of $i = 1.9$.

We were surprised to find the minimum wall time of the solver at such a high exponent, since we expected $i = 1$ for an efficient solver. This may be related to the implementation of the preconditioner combined with the disorder that \hp-adaptive methods cause in the system matrix. Further, we were expecting a minimum in the assembly at about $i = 2$, but found it was decaying at even higher exponents. We have no explanation for this behavior. Analyzing the effect of cell weighting on each individual section of the program
%the load balancing
will be subject of further investigations.

% Next step
%We set an weighting expoenent to a value of $i = 1.9$ for the following considerations regarding scalability.
