\chapter{Application on Laplace's problem}
\label{ch:results}
\glsresetall

%Case 0: step-27
%\begin{align}
%  \Omega &= [-1,1]^d \setminus \left(-\frac{1}{2}, \frac{1}{2}\right)^d ,\\
%  - \nabla^2 u(\vec{x}) &= \prod\limits_{j=1}^{d} (x_j + 1) \quad\text{on}\quad \Omega,\\
%  u(\vec{x}) &= 0 \quad\text{on}\quad \partial\Omega
%\end{align}
%for $u$ scalar solution and $\vec{x} = (x_1, x_2, \dots, x_d)$ vector with $d$-components.

With all algorithms and data structures elaborated in Chs.~\ref{ch:parallel} and \ref{ch:dynamic}, we now have all features at our disposal to solve partial differential equations with parallel, dynamic \hp-adaptive \glspl{fem}. As a next step, we would like to apply our implementation in the \dealii{} library on a certain exemplary problem that showcases its error performance and parallel scalability. To quantify its capabilities, we want to find a scenario for which an analytic solution $u_\text{sol}$ is available allowing us to calculate the actual error of the finite element approximation. This approach for code verification corresponds to the method of manufactured solutions \parencite{salari2000}. Our choice for a suitable problem falls on solving the Laplace problem with Dirichlet boundary conditions:
\begin{align}
- \nabla^2 u(\vec{x}) &= 0 \quad\text{on}\quad \Omega \,\text{,} & u(\vec{x}) &= u_\text{sol}(\vec{x}) \quad\text{on}\quad \partial\Omega \,\text{.}
\end{align}

Following Eq.~\ref{eq:errorbound_hp}, we are aware that \p-adaptation is favorable in regions where the solution is regular, while \h-adaptation yields better results in regions with discontinuities or singularities. For elliptic problems like the Laplace one, we expect singularities on concave domains \parencite[Sec.~5.5]{brenner2008}.

\textcite{mitchell2014} presented several benchmarks for \hp-adaptation that make use of this observation. We decide to showcase our implementation on a two-dimensional domain with a reentrant corner located at the point of origin:
\begin{align}
\Omega &= \left\{ (r,\varphi) : (0 \leq r) \wedge( 0 \leq \varphi \leq \pi/\alpha) \right\} \,\text{,}
\end{align}
with $\alpha \in \left(1/2, 1\right)$. With the additional requirement that the solution must be zero along the legs of the reentrant corner, this particular scenario has a general solution which can be formulated in polar coordinates $r = \sqrt{x^2 + y^2} > 0$ and $\theta = \arctantwo(y,x)$:
\begin{align}
u_\text{sol}(\vec{x}) &= r^\alpha \sin(\alpha \, \theta) \,\text{,} \\
\nonumber \nabla u_\text{sol}(\vec{x}) &= \partial_r u_\text{sol}(\vec{x}) \vec{e}_r + \frac{1}{r} \partial_\theta u_\text{sol}(\vec{x}) \vec{e}_\theta \\
&= \alpha r^{\alpha - 1} \left[ \sin(\alpha \, \theta) \vec{e}_r + \cos(\alpha \, \theta) \vec{e}_\theta \right] \,\text{,}
\end{align}
with unit vectors \(\vec{e}_r = \cos(\theta) \vec{e}_x - \sin(\theta) \vec{e}_y\) and \(\vec{e}_\theta = \sin(\theta) \vec{e}_x + \cos(\theta) \vec{e}_y\). We immediately see that this solution has a singularity near the point of origin for the permitted values of $\alpha \in \left(1/2, 1\right)$:
\begin{align}
\left\| \nabla u_\text{sol}(\vec{x}) \right\|_{2} &= \alpha r^{\alpha - 1} \,\text{,} & \lim\limits_{r \rightarrow 0} \left\| \nabla u_\text{sol}(\vec{x}) \right\|_{2} &= \infty \quad\text{for}\quad \alpha \in \left(1/2, 1\right) \,\text{.}
\end{align}

In our testcase, we pick the corner to have a right angle with $\alpha = 2/3$ resulting in an L-shaped domain, at which all cells share the same topology to exclude influences from mesh distortion in our benchmark:
\begin{align}
\Omega_\text{L} &= \left[-1,1\right]^2 \setminus \left\{ \left(0,1\right) \times \left(-1, 0\right) \right\} \,\text{.}
\end{align}
A depiction of the solution on this particular domain is shown in Fig.~\ref{fig:solution}. The three dimensional variant of this problem is often referred to as the Fichera corner problem.

\begin{figure}
\centering
\includegraphics[width=0.5\textwidth]{figures/results/solution.png}
\caption{Solution of the manufactured Laplace problem on a L-shaped domain.}
\label{fig:solution}
\end{figure}

In this chapter, we will solve the so designed Laplace problem on the L-shaped domain on consecutively refined meshes, and evaluate certain aspects of our implementation, namely the error performance of the decision strategies and their parallel scalability.

Following the usual approach in \gls{fem}, we transfer our problem to its weak formulation using \gls{cg} methods \parencite{brenner2008}:
\begin{align}
\left(\nabla u, \nabla v\right) &= 0 \,\text{,} & \forall v \in V &\coloneqq \left\{ v \in H^1(\Omega): v|_{\partial\Omega} = 0 \right\} \,\text{.}
\end{align}
The shape functions of Lagrangian elements will form the basis for the function space $V$. Dirichlet boundary conditions are imposed via constraints on \glspl{dof} located on boundaries. The problem will be solved numerically with an iterative solver based on the conjugate gradient algorithm combined with an \gls{amg} preconditioner.

The \dealii{} library offers interfaces to parallel linear algebra of the third party libraries \petsc{} \textcite{petsc3124} and \trilinos{} \textcite{trilinos12181} for distributed memory architectures. In our investigations, we choose the latter using their \epetra{}
%\textcite{epetra}
module that handles all data infrastructure, and pick a corresponding solver from their \aztecoo{}
%\textcite{aztecoo}
package as well as their \ml{}
%\textcite{ml}
preconditioner. Compared to an equivalent configuration of \petsc{} modules, the \trilinos{} implementation yields more reproducible results using \gls{mpi} \parencite[FAQ]{petsc3124} and performs faster with higher order polynomials at more advanced refinement iterations according to our experience. For all calculations, we set a tolerance of $10^{-12}$ relative to the $l_2$-norm of the right hand side vector of the equation system.



\section{Error versus performance}
\label{sec:errorvsperformance}

Text.

\begin{figure}
\centering
\begin{tikzpicture}
\begin{loglogaxis}[
  xlabel=Number of \glspl{dof},
  ylabel=H1 error]
\addplot table [y=H1 error, x=ndofs, col sep=comma] {data/error/h.csv};
\addplot table [y=H1 error, x=ndofs, col sep=comma] {data/error/p.csv};
\addplot table [y=H1 error, x=ndofs, col sep=comma] {data/error/hp-legendre.csv};
\addplot table [y=H1 error, x=ndofs, col sep=comma] {data/error/hp-fourier.csv};
\addplot table [y=H1 error, x=ndofs, col sep=comma] {data/error/hp-prediction.csv};
\addplot table [y=H1 error, x=ndofs, col sep=comma] {data/error/hp-legendre-naive.csv};
\addplot table [y=H1 error, x=ndofs, col sep=comma] {data/error/hp-fourier-naive.csv};
\addplot table [y=H1 error, x=ndofs, col sep=comma] {data/error/hp-prediction-naive.csv};
\end{loglogaxis}
\end{tikzpicture}
\caption{Error vs workload.}
\label{fig:errorworkload}
\end{figure}

\begin{figure}
\centering
\begin{tikzpicture}
\begin{loglogaxis}[
  xlabel=Wall time {[seconds]},
  ylabel=H1 error,
  legend pos=outer north east]
\addplot table [y=H1 error, x=walltime, col sep=comma] {data/error/h.csv};
\addlegendentry{\h};

\addplot table [y=H1 error, x=walltime, col sep=comma] {data/error/p.csv};
\addlegendentry{\p};

\addplot table [y=H1 error, x=walltime, col sep=comma] {data/error/hp-legendre.csv};
\addlegendentry{\hp{} Legendre};

\addplot table [y=H1 error, x=walltime, col sep=comma] {data/error/hp-fourier.csv};
\addlegendentry{\hp{} Fourier};

\addplot table [y=H1 error, x=walltime, col sep=comma] {data/error/hp-prediction.csv};
\addlegendentry{\hp{} Prediction};

\addplot table [y=H1 error, x=walltime, col sep=comma] {data/error/hp-legendre-naive.csv};
\addlegendentry{\hp{} Legendre naive};

\addplot table [y=H1 error, x=walltime, col sep=comma] {data/error/hp-fourier-naive.csv};
\addlegendentry{\hp{} Fourier naive};

\addplot table [y=H1 error, x=walltime, col sep=comma] {data/error/hp-prediction-naive.csv};
\addlegendentry{\hp{} Prediction naive};
\end{loglogaxis}
\end{tikzpicture}
\caption{Error vs walltime.}
\label{fig:errorwalltime}
\end{figure}
\section{Load balancing heuristics}
\label{sec:heuristics}

Text.

\todo{Second scale for assembly}
\begin{figure}
\centering
\begin{tikzpicture}
\begin{axis}[
  xlabel=Weighting exponent,
  ylabel=Walltime {[seconds]}]

\addplot table [y=full cycle, x=weighting exponent, col sep=comma] {data/weight/weight.csv};
\addplot table [y=assembly, x=weighting exponent, col sep=comma] {data/weight/weight.csv};
\addplot table [y=solve, x=weighting exponent, col sep=comma] {data/weight/weight.csv};
\end{axis}
\end{tikzpicture}
\caption{Load balancing.}
\label{fig:weights}
\end{figure}
\section{\Glsfmtlong{hpc} scalability}
\label{sec:scaling}

Text.

\begin{figure}
\begin{subfigure}{1\textwidth}
  \centering
  \begin{tikzpicture}
\begin{loglogaxis}[
  xlabel=Number of \glspl{dof},
  ylabel=Walltime {[seconds]}]

\addplot table [y=full cycle, x=ndofs, col sep=comma] {data/weak/weak-nodes16.csv};

\addplot[very thick, samples=2, domain=12591105:1302365268] {10^(-3)*x};

\end{loglogaxis}
\end{tikzpicture}
  \caption{Weak scaling on 16 nodes or 768 \gls{mpi} processes.}
  \label{fig:weak-nodes16}
\end{subfigure}
\begin{subfigure}{1\textwidth}
  \centering
  \begin{tikzpicture}
\begin{loglogaxis}[
  xlabel=Number of \glspl{dof},
  ylabel=Wall time {[seconds]},
  legend pos=outer north east]

% auxiliary lines
\begin{scope}
  \draw[green] ({axis cs:307200000,0}|-{rel axis cs:0,1}) -- ({axis cs:307200000,0}|-{rel axis cs:0,0});
  \draw[blue] ({axis cs:51625521,0}|-{rel axis cs:0,1}) -- ({axis cs:51625521,0}|-{rel axis cs:0,0});
\end{scope}

% data
\addplot table [y=solve, x=ndofs, col sep=comma] {data/weak/weak-nodes64.csv};
\addlegendentry{linear solver and preconditioner};

\addplot table [y=setup, x=ndofs, col sep=comma] {data/weak/weak-nodes64.csv};
\addlegendentry{setup data structures};

\addplot table [y=assembly, x=ndofs, col sep=comma] {data/weak/weak-nodes64.csv};
\addlegendentry{assemble linear system};

\addplot table [y=compute errors, x=ndofs, col sep=comma] {data/weak/weak-nodes64.csv};
\addlegendentry{estimate error};

\addplot table [y=calculate indicators, x=ndofs, col sep=comma] {data/weak/weak-nodes64.csv};
\addlegendentry{estimate smoothness};

\addplot table [y=refine, x=ndofs, col sep=comma] {data/weak/weak-nodes64.csv};
\addlegendentry{coarsen and refine};

% optimal line
\addplot[very thick, samples=2, domain=12591105:2073075769] {10^(-7.2)*x};

\end{loglogaxis}
\end{tikzpicture}
  \caption{Weak scaling on 64 nodes or 3,072 \gls{mpi} processes.}
  \label{fig:weak-nodes64}
\end{subfigure}
\caption{Weak scaling.}
\label{fig:weak}
\end{figure}

\todo{Add remaining data points}
\begin{figure}
\begin{subfigure}{1\textwidth}
  \centering
  \begin{tikzpicture}
\begin{loglogaxis}[
  xlabel=Number of \gls{mpi} processes,
  ylabel=Walltime {[seconds]}]

\addplot table [y=full cycle, x=ncpus, col sep=comma] {data/strong/strong-nrefs10_withoutlarge.csv};

\addplot[very thick, samples=2, domain=48:6144] {10^(3)*x^(-1)};

\end{loglogaxis}
\end{tikzpicture}
  \caption{Strong scaling with a fixed problem size of roughly 50 million \glspl{dof}.}
  \label{fig:strong-nrefs10}
\end{subfigure}
\begin{subfigure}{1\textwidth}
  \centering
  \begin{tikzpicture}
\begin{loglogaxis}[
  xlabel=Number of \gls{mpi} processes,
  ylabel=Wall time {[seconds]},
  legend pos=outer north east]

% auxiliary lines
\begin{scope}
  \draw[green] ({axis cs:9692.57276,0}|-{rel axis cs:0,1}) -- ({axis cs:9692.57276,0}|-{rel axis cs:0,0});
\end{scope}

% data
\addplot table [y=solve, x=ncpus, col sep=comma] {data/strong/strong-nrefs12_withoutlarge.csv};
\addlegendentry{linear solver and preconditioner};

\addplot table [y=setup, x=ncpus, col sep=comma] {data/strong/strong-nrefs12_withoutlarge.csv};
\addlegendentry{setup data structures};

\addplot table [y=assembly, x=ncpus, col sep=comma] {data/strong/strong-nrefs12_withoutlarge.csv};
\addlegendentry{assemble linear system};

\addplot table [y=compute errors, x=ncpus, col sep=comma] {data/strong/strong-nrefs12_withoutlarge.csv};
\addlegendentry{estimate error};

\addplot table [y=calculate indicators, x=ncpus, col sep=comma] {data/strong/strong-nrefs12_withoutlarge.csv};
\addlegendentry{estimate smoothness};

\addplot table [y=refine, x=ncpus, col sep=comma] {data/strong/strong-nrefs12_withoutlarge.csv};
\addlegendentry{coarsen and refine};

% optimal line
\addplot[very thick, samples=2, domain=768:49152] {10^(5)*x^(-1)};

\end{loglogaxis}
\end{tikzpicture}
  \caption{Strong scaling with a fixed problem size of roughly 900 million \glspl{dof}.}
  \label{fig:strong-nrefs12}
\end{subfigure}
\caption{Strong scaling.}
\label{fig:strong}
\end{figure}