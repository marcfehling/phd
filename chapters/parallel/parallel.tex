\chapter{Massively parallel \hp-adaptive \glsfmtlongpl{fem}}
\label{ch:parallel}
\glsresetall

Any kind of numerical method requires thorough thought on designing suitable algorithms and data structures with respect to correctness, robustness and performance. In general, the ideas behind implementations of these methods are similar and can be generalized.
%, especially when they are based on the same numerical method.
This is also the case for additional enhancements that improve basic realizations. In case of the \gls{fem} such features are, for example, parallelisation, adaptive methods, continuous or discontinuous Galerkin methods, and the support for complex geometries.

In this chapter, we will present algorithms and data structures for parallel, \hp-adaptive \glsfmtlongpl{fem}. Generalized thoughts on each of the two aspects have already been presented: \textcite{bangerth2009} developed a general implementation for \hp-adaptive \gls{fem} software; and a generalized distributed computing approach of the \glsfmtlong{fem} has been introduced by \textcite{bangerth2012}. However, the consolidation of both features is not trivial. We will elaborate on the trickiest facets in the following sections, after providing a brief introduction to \gls{fem} and the parallelisation approach we will follow.

Thus chapter should be understood as an enhancement of the two aforementioned contributions and bases on large parts on it. We recommend a previous reading of both articles.


\subsubsection{Basics of the \glsfmtlong{fem}}

The basic idea of the \glsfmtlong{fem} conforms to the specification of a function space and finding a solution to the investigated problem in it. To be more precise, we pick a suitable set of basis functions for which the solution is a linear combination. Its coefficients are called unknowns or \glspl{dof}, since their values are determined after solving the problem.

In general, \gls{fem} requires a subdivision of the domain into smaller cells, typically into triangles or quadrilaterals in two and tetrahedra or hexahedra in three dimensions. All these cells are mappings of a reference cell, to which we assign a set of shape functions with corresponding support points.
%Those functions are designed to have the value one on exactly one support point and the value zero on all others.
The composition of reference cell, shape functions and support points is defined as a finite element.

The collection of all cells with their corresponding finite elements assigned form the aforementioned function space. In the global mesh, each support point for each finite element will be uniquely identified with a \gls{dof} on each cell. Consulting the weak formulation of the investigated problem, we are then able to formulate and solve the corresponding system of linear equations.

\todo{add some equations and math symbols}
$u_h(\vec{x}) = \sum_{j=0}^{N-1} U_j \varphi_j(\vec{x})$
% $u_h \in V_h$

This is just a brief introduction to \gls{fem} in order to provide the fundamentals for this chapter and to specify the nomenclature used. More details on this topic can be found in common literature \parencite[e.g.][]{girault1986, elman2014, kuzmin2015} \todo{add more literature?}. Especially \textcite{brenner2008} provided a more rigorous and mathematically sound definition on finite elements.


\subsubsection{Parallelization approach for adaptive meshes}

For distributed computing.

This involves a hierarchy

However, to stay flexible during 

This is just a brief outline of all the requirements that \textcite{bangerth2012} worked out, which is crucial for the upcoming section.


\section{Enumeration of \glsfmtlongpl{dof}}
\label{sec:enumeration}

Text.
\section{Data transfer}
\label{sec:transfer}

Text.

Without \p-adaptivity, the number of \glspl{dof} or rather the amount of data to store per cell does not differ. Thus, we know how much data to send or receive on each cell. This is no longer the case with \p-adaptivity.
\section{Load balancing}
\label{sec:balancing}

The efficient use of all computational resources requires a uniform distribution of all workload among them. There are many factors that determine the workload in a \gls{fem} application, above all the preparation of data structures, the assembly of both the matrix and right hand side of the linear equation system, and the choice of the type of its solver.

In most \h-adaptive applications, cells are similar and correspond to the same workload. Thus, we can simply balance the number of cells on all processes. However with \hp-adaptive application, this is no longer the case due to the diversity of finite elements. %Here, the workload varies with the number of \glspl{dof} on each cell. (and the quadrature)
In this case, since the domain is partitioned on the basis of cells, we need to assign a corresponding weight to every cell that determines its individual workload and balance the cumulated weights among all processes.

The workload of each cell depends on its assigned finite element and quadrature formula, and correlates to the number of \glspl{dof} and quadrature points, among other quantifiable values that depend on the individual problem. For example, Lagrangian elements of different order as depicted in Fig.~\ref{fig:lagrange} each have a distinct number of \glspl{dof} assigned.

%For example, consider Lagrangian elements of different polynomial degrees with different number of \glspl{dof} depicted in Fig.~\ref{fig:lagrange} require a different workload, mainly consumed by matrix assembly and the solution process.

%and can be quantified
%We expect a correlation of every cell's individual workload with the assigned number of \glspl{dof} and number of quadrature point of the assigned formula, among others quantifiable values.

%Ideally, we want to balance the workload. We will thus perform weighting.

%and balance the cumulated sum of cells to be equal on all processes. We can do that with a prefix sum.

For the purpose of load balancing, \textcite[Sec.~3.3]{burstedde2011} provided an algorithm for weighted partitioning and enhanced \pforest{} \textcite{p4est22} with a corresponding implementation, from which we take advantage in \dealii{}. Omitting details about the communication between processes, we will briefly outline its basic idea: On a distributed mesh, calculate the prefix sum of cell weights in the global scope, determine the partition boundaries with a binary search, and transfer cells to their new owning processes if necessary.

%The partitioning algorithm of \pforest{}

%\pforest{} offers such a weighting mechanism. The basic idea behind it is, that we calculate the cumulated sum of cells with a prefix sum, and assigning the correpsonding partition to our process with a binary search.

%We rely on the implementation of \textcite{burstedde2018} in \pforest{} \textcite{p4est22}. The basic idea of this algorithm is to form a cumulated of all cell weights with an \texttt{MPI\_Exscan} call, and then each process will find its balanced range of cells with binary searches.

In the context of \hp-adaptive \gls{fem} applications, we identify the assembly of the linear equation system and its solution as the most expensive tasks, and correlate their individual contribution to the workload on each cell with the number of \glspl{dof} from the assigned finite element.
%As a first indicator for the workload, we make the number of \glspl{dof} responsible.
The workload of efficient solvers scales with the number of \glspl{dof} $\mathcal{O}\left(n_\text{dofs}\right)$, while we suppose that the workload of the assembly will be of order $\mathcal{O}\left(n_\text{dofs}^2\right)$ since we fill quadratic matrices.

We expect that the actual workload of an \hp-adaptive \gls{fem} application will actually scale with an order somewhere in between the two, i.e.\@ $\mathcal{O}\left(n_\text{dofs}^c\right)$ with a constant exponent $c \in [1,2]$. We use this strategy for investigations in this dissertation in which we also determine a suitable exponent $c$.

Alternatively, weighting each cell with a factor of $(a \, n_\text{dofs}^2 + b \, n_\text{dofs})$ appears conceivable, for which the partitioning results depend on the ratio of both constants $a$ and $b$.

%We will assign a weight $n_\text{dofs}^i$ to every cell, % and balance the cumulated sum of cells to be equal on all processes. We can do that with a prefix sum.

A reliable measure of weights is tied to the type of problem and the choice of the solver. With the presented approach, we still have to specify a suitable weight manually. It is part of future work to supply a heuristic from which we will determine suitable weights automatically. %, depending on the type of problem and the choice of the solver.